\section{Aussagenlogik}

Junktoren: \\
nicht, Negation $\neg$ \\
und, Konjunktion $\wedge$ \\
oder, Disjunktion $\vee$ \\
ausschließendes oder, Kontravalenz $\asymp$ \\
Subjunktion, Implikation $\rightarrow$ \\
"wenn $\cdots$ dann" \\
\[
\begin{array}{c c | c }
	A & B & A \rightarrow B \\ \hline
	1 & 1 & 1 \\
	1 & 0 & 0 \\
	0 & 1 & 1 \\
	0 & 0 & 1
\end{array}
\]\\
Bijunktion, $\leftrightarrow$
"genau dann $\cdots$ wenn"
\[
\begin{array}{c c | c }
	A & B & A \leftrightarrow B \\ \hline
	1 & 1 & 1 \\
	1 & 0 & 0 \\
	0 & 1 & 0 \\
	0 & 0 & 1
\end{array}
\]\\

Rechenregeln: \\
$\neg$ bindet am stärksten.\\
$ (\neg A) \wedge B \leftrightarrow \neg A \wedge B $ \\


$\vee$ und $\wedge$ binden stärker als $\rightarrow$ und $\leftrightarrow$ \\
$ A \wedge B \leftrightarrow C \equiv (A \wedge B) \leftrightarrow C $ \\

$\vee$ und $\wedge$ sind gleichwertig\\
$ (A \wedge B) \vee C \not\equiv A \wedge B \vee C $ \\

$\rightarrow$ und $\leftrightarrow$ sind gleichwertig\\
$ (A \rightarrow B) \leftrightarrow C \not\equiv A \rightarrow B \leftrightarrow C $ \\

Äquivalenz (Logisch): \\
$\Leftrightarrow$ \\
Drückt eine Tautologie einer Bijunktion aus.
Dies tritt auf, wenn die Werte einer Bijunktion alle wahr sind.


\section{Prädikatenlogik}

Prädikat: \\
"Wort-Folge" mit $n$ Leerstellen, die zur Aussage wird,
wenn ein Idividuum eingesetzt wird. \\

Individuum: \\
"Eigenname": Element für das die Aussage gilt. \\

Prädikat $P$ = "größer als 0 \\
Individuum $i$ = x \\
$P(i)$\\

Prädikat mit mehreren Individuen $P(a,b)$\\

Allquantor: \\



\section{Beweismethoden}

\subsection{Direkter Beweis}

\subsection{Indirekter Beweis}

\subsubsection{Beweis durch Kontraposition}

\subsubsection{Beweis durch Widerspruch}

\subsection{Vollständige Induktion}